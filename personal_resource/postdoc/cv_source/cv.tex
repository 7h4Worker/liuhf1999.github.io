\documentclass[11pt,a4paper]{article}
\usepackage{geometry}
\geometry{top=1.5cm,bottom=1.8cm,left=1.6cm,right=1.6cm}
\usepackage{fontspec}
\usepackage{xeCJK}
\setmainfont{Times New Roman}
\setsansfont{Arial}
\setmonofont{Consolas}
\setCJKmainfont{SimSun}
\usepackage{enumitem}
\usepackage{array}
\usepackage{xcolor}
\usepackage{hyperref}
\hypersetup{colorlinks=true,urlcolor=Primary}
\usepackage{titlesec}

\definecolor{Primary}{RGB}{0,128,170}
\setlength{\parindent}{0pt}
\setlength{\parskip}{0.3em}
\setlist[itemize]{leftmargin=1.4em,itemsep=0.2em,topsep=0.2em}

\newcommand{\sectionnote}[2]{%
  \vspace{1.3em}%
  {\Large\bfseries\textcolor{Primary}{#1}}\hfill{\small\textcolor{Primary}{#2}}\par
  \textcolor{Primary}{\rule{\textwidth}{0.9pt}}\vspace{0.4em}}

\begin{document}
\pagestyle{empty}

\begin{center}
  {\Huge\bfseries\textcolor{Primary}{HAIFENG LIU}}\\
\end{center}

\sectionnote{PERSONAL INFORMATION}{(这部分可以把自己的照片放上)}
\begin{tabular}{@{}p{0.23\textwidth}p{0.27\textwidth}p{0.23\textwidth}p{0.27\textwidth}@{}}
\textbf{Gender:} Male & \textbf{Birth:} 1998.XX(可更新) & \textbf{Marital Status:} Single(可根据需要补充) &\\
\textbf{Nationality:} Chinese & \textbf{Email:} \href{mailto:liuhf@shanghaitech.edu.cn}{liuhf@shanghaitech.edu.cn} & \textbf{Address:} 上海市浦东新区张江高科技园区海科路99号 & \textbf{Telephone:} 173-4201-8052
\end{tabular}

\medskip
\textbf{Profile:} 面向双目立体视觉特性的脑机接口范式、算法与系统,关注沉浸式 VR/AR 场景下的高效可穿戴 BCI。结合自研 EEG 硬件、FPGA 加速与多模态解码,为下一代脑机交互产品提供从算法到工程的解决方案。

\sectionnote{EDUCATION}{}
\textbf{Ph.D. Candidate: ShanghaiTech University — 2021.9-Present (Expected graduation: Jul 2026)}\\
Major in Computer Science and Technology. CAS 联合培养博士生。\\
Supervised by Prof. Honglin Hu. 研究方向包括 VR/AR 脑机接口、稳态视觉诱发电位(SSVEP)调制、脑对脑接口与多模态信号处理。\\
PhD Thesis (计划主题): Binocular-encoded VR-BCI Paradigms with Ultra-flexible Acquisition Platforms.\\
Courses: Neural Signal Processing, Machine Learning, Advanced Virtual Reality Systems.

\medskip
\textbf{Bachelor Degree: Zhejiang University — 2017.9-2021.7}\\
Major in Information Engineering. GPA: (可补充)\\
Final year laboratory project: 融合式脑电采集平台设计与实现。\\
Courses: 信号与系统、数字电路、嵌入式系统、控制工程等。

\sectionnote{RESEARCH SKILLS}{}
\textbf{Language Proficiency:} English: passed CET6(620+),具备学术写作与国际会议交流能力。

\textbf{Laboratory Techniques:}\\
高密度 EEG/EMG/ECG 采集系统设计,ADS1299 / Nordics / ESP32 硬件开发;立体视觉刺激设计;脑信号实验流程与数据质量控制;VR/AR 设备联调;超柔性电极阵列实验。

\textbf{Digital Skills:}\\
MATLAB、Python(NumPy、SciPy、PyTorch)用于信号处理与深度学习解码;C/C++、C\# 与 Unity 搭建交互系统;FPGA(Xilinx)、Zephyr RTOS、Fusion360 CAD、LabVIEW/Vive Coding 可视化编程等。

\sectionnote{RESEARCH \& PROJECT EXPERIENCES}{}
\textbf{SariBCI 脑电信号采集系统} \hfill 2022.7--2022.11\\
基于 ADS1299 打造 24-bit/250--1000 Hz 的高精度可穿戴 EEG 设备,支持 WiFi AP/Station 模式;开发 Unity 双视角 3D 刺激界面与实时分析工具,结合 3D/3D-Blink 范式显著提升 SSVEP 识别率。

\textbf{Hololens 双目 AR-BCI 闭环} \hfill 2023.5--2023.9\\
搭建面向 Hololens 的 UWP 应用,将 SariBCI 数据流接入 VR/AR 平台;集成 Zigbee 外设控制端,实现脑控多手势交互与移动部署,展示多模态融合能力。

\textbf{多模态多传感器信号采集一体化} \hfill 2024.3--2025.7\\
整合 EEG/ECG/EMG/IMU 传感终端与上位机协同推理;提出时序+空间特征的在线深度学习模型,并在 FPGA 上部署嵌入式推理,支持低延迟脑机交互。

\textbf{超宽带脑电采集与远程传输系统} \hfill 2024.6--2025.9\\
设计多通道时域压缩与超宽带高速采集方案,构建低功耗无线脑电平台;完成云端实时传输与管理,实现跨地点脑信号协作实验。

\sectionnote{PUBLICATIONS}{}
\begin{enumerate}[leftmargin=1.5em,label={[}\arabic*{]}]
  \item \textbf{H.~Liu}, Z. Wang, R. Li, X. Zhao, T. Xu, T. Zhou, H. Hu, ``A Comparative Study of Stereo-Dependent SSVEP Targets and Their Impact on VR-BCI Performance,'' \textit{Frontiers in Neuroscience}, 18, 1367932, 2024. doi: \href{https://doi.org/10.3389/fnins.2024.1367932}{10.3389/fnins.2024.1367932}
  \item \textbf{H.~Liu}, Z. Wang, R. Li, et al., ``A Novel SSVEP Modulation Method Utilizing VR-Based Binocular Vision,'' \textit{IEEE EMBC}, 2024, pp.~1--4. doi: \href{https://doi.org/10.1109/EMBC53108.2024.10781783}{10.1109/EMBC53108.2024.10781783}
  \item \textbf{H.~Liu}, Z. Zhu, Z. Wang, et al., ``Design and Implementation of a Scalable and High-Throughput EEG Acquisition and Analysis System,'' \textit{Moore and More}, 1, 14, 2024. doi: \href{https://doi.org/10.1007/s44275-024-00017-w}{10.1007/s44275-024-00017-w}
  \item Y. Ye, X. Tian, \textbf{H.~Liu}, et al., ``High-Precision, Low-Threshold Neuromodulation With Ultraflexible Electrode Arrays for Brain-to-Brain Interfaces,'' \textit{Exploration}, e70040, 2025. doi: \href{https://doi.org/10.1002/EXP.70040}{10.1002/EXP.70040}
  \item A. Li, \textbf{H.~Liu}, et al., ``Enhancing Real-Time Online Motor Imagery BCI Performance: A Co-Adaptive Meta-Learning Approach,'' \textit{IEEE Journal of Biomedical and Health Informatics}, revision in progress. doi: 待定。
  \item \textbf{H.~Liu}, Z. Wang, R. Li, X. Zhao, T. Xu, T. Zhou, H. Hu, ``A Novel Binocular-Encoded SSVEP Framework for Efficient VR-Based Brain-Computer Interface,'' \textit{IEEE Journal of Biomedical and Health Informatics}, accepted Oct 2025. doi: 待定。
\end{enumerate}

\textbf{Patents:}\\
【国家发明专利】胡宏松,\textbf{刘海峰} 等,“基于 SSVEP-ERP 的脑机接口视觉范式生成、检测方法、系统、介质、终端”,CN202310369762.8(已公开)。\\
【国家发明专利】胡宏松,\textbf{刘海峰} 等,“一种基于增强现实和脑机接口的新型脑反馈系统”,202410293427.9(已初审)。

\sectionnote{CONFERENCE PROCEEDINGS}{}
\begin{itemize}
  \item 2024 IEEE Engineering in Medicine and Biology Conference (EMBC), Orlando, USA — Oral presentation on VR-based SSVEP modulation.
  \item 2023 Shanghai VR/AR Frontier Forum — Poster on binocular BCI fusion architecture.
\end{itemize}

\sectionnote{SCHOLARSHIPS AND AWARDS}{}
\begin{itemize}
  \item 上海科技大学研究生学业奖学金(可补充具体等级与年份)。
  \item 国家/省部级竞赛或科研奖项(如国家奖学金、三好学生、优秀研究生等,可在此处填写)。
\end{itemize}

\sectionnote{TEACHING \& VOLUNTEER EXPERIENCES}{}
\begin{itemize}
  \item 课程助教经历(如《脑机接口导论》《信号处理》),可写明年份与职责。
  \item 科研指导或志愿服务:例如脑机接口科普讲座、科技竞赛辅导、社区 STEM 志愿活动等。
\end{itemize}

\end{document}
