\documentclass[11pt,a4paper]{article}
\usepackage{geometry}
\geometry{top=1.5cm,bottom=1.8cm,left=1.6cm,right=1.6cm}
\usepackage{fontspec}
\usepackage{xeCJK}
\usepackage{graphicx}
\usepackage{enumitem}
\usepackage{xcolor}
\usepackage{hyperref}
\usepackage{array}

\setmainfont{Times New Roman}
\setsansfont{Arial}
\setmonofont{Consolas}
\setCJKmainfont{SimSun}

\definecolor{Primary}{RGB}{0,118,163}
\definecolor{Accent}{RGB}{190,209,224}

\hypersetup{
  colorlinks=true,
  linkcolor=Primary,
  urlcolor=Primary
}

\setlength{\parindent}{0pt}
\setlength{\parskip}{0.3em}
\setlist[itemize]{leftmargin=1.4em,itemsep=0.3em,topsep=0.2em}

\newcommand{\sectiontitle}[1]{%
  \vspace{1.2em}%
  {\Large\bfseries\textcolor{Primary}{#1}}\par
  \textcolor{Primary}{\rule{\textwidth}{0.8pt}}\vspace{0.3em}
}

\newcommand{\timeline}[2]{%
  \textbf{#1}\quad #2\par
}


\begin{document}
\pagestyle{empty}

\begin{minipage}[t]{0.72\textwidth}
\textbf{Name:} Haifeng Liu\\
\textbf{Phone:} +86-173-4201-8052\\
\textbf{Email:} \href{mailto:liuhf@shanghaitech.edu.cn}{liuhf@shanghaitech.edu.cn}\\
\textbf{Political Status:} CPC Member\\
\textbf{Address:} 99 Haike Road, Zhangjiang Hi-Tech Park, Pudong, Shanghai, China
\end{minipage}
\begin{minipage}[t]{0.26\textwidth}
\raggedleft
\fbox{\parbox[c][4.4cm][c]{3.5cm}{\centering Photo}}\\
{\small Replace with a headshot}
\end{minipage}

\sectiontitle{Education}
\begin{tabular}{@{}p{0.23\textwidth}p{0.42\textwidth}p{0.3\textwidth}@{}}
2017.9--2021.7 & Zhejiang University & B.Eng. in Information Engineering\\
2021.9--2026.7 & ShanghaiTech University (Joint Ph.D. Program with CAS) & Ph.D. Candidate in Computer Science and Technology (expected Jul 2026)
\end{tabular}

\sectiontitle{Research Focus}
\textbf{Overview:} Brain--computer interface paradigms and algorithms that exploit binocular vision cues to enable efficient wearable BCIs tightly integrated with VR/AR headsets.
\begin{itemize}
  \item Conduct comparative analyses of stereo-dependent SSVEP targets to quantify their impact on VR BCI performance.
  \item Designed 3D and 3D-Blink paradigms that leverage stereoscopic comfort to enhance user experience and system robustness in immersive environments.
\end{itemize}

\timeline{2024.7}{Novel binocular steady-state visual evoked potential (SSVEP) modulation}
\begin{itemize}
  \item Developed a VR dual-view SSVEP modulation approach that combines left/right view layers to encode nine targets with only two frequencies, significantly improving robustness.
  \item Implemented the FusionCA neural-fusion interface platform, establishing multi-channel information pathways for binocular headsets and richer immersion.
\end{itemize}

\timeline{2024.9}{Scalable high-density EEG acquisition and analytics platform}
\begin{itemize}
  \item Built an FPGA-based modular EEG acquisition system providing standardized hardware interfaces suitable for laboratory BCI studies and clinical scenarios.
\end{itemize}

\timeline{2025.1}{High-precision neuromodulation with ultraflexible electrode arrays}
\begin{itemize}
  \item Created a brain-to-brain interface that routes human BCI commands to rodent locomotion control through ultraflexible electrode arrays.
  \item Established online transmission and DNN decoding pipelines to map human-side BCI commands within bio-inspired systems.
\end{itemize}

\sectiontitle{Project Experience}
\timeline{2022.7--2022.11}{SariBCI EEG acquisition system}
\begin{itemize}
  \item Designed a high-precision ADS1299-based EEG module achieving 24-bit resolution and 250--1000 Hz sampling, with WiFi AP/Station modes and synchronized multi-channel acquisition.
  \item Built Unity-based dual-view 3D cue interfaces and analytics for SSVEP paradigms, incorporating 3D/3D-Blink stimuli for improved efficacy.
\end{itemize}

\timeline{2023.5--2023.9}{Hololens-oriented AR-BCI closed-loop platform}
\begin{itemize}
  \item Developed UWP applications for Hololens and other headsets, streamlining the SariBCI data pipeline for VR/AR closed-loop control.
  \item Integrated Zigbee peripherals to extend multi-sensor nodes, supporting brain-controlled multi-gesture interaction and mobile deployment.
\end{itemize}

\timeline{2024.3--2025.7}{Multimodal sensing and data fusion}
\begin{itemize}
  \item Engineered sensor terminals combining EEG, ECG, EMG, and IMU signals; coordinated hardware cooperation through host-side inference to realize multimodal fusion.
  \item Proposed online deep-learning models capturing temporal and spatiotemporal features, deploying them onto FPGA platforms for embedded edge inference.
\end{itemize}

\timeline{2024.6--2025.9}{Ultra-wideband EEG acquisition and remote streaming}
\begin{itemize}
  \item Introduced multi-channel time-domain compression with ultra-wideband high-speed acquisition, enabling low-power wireless EEG across heterogeneous sensors.
  \item Completed multi-channel porting and real-time cloud interaction to form a long-range EEG transmission and management infrastructure.
\end{itemize}

\sectiontitle{Publications}
\begin{enumerate}[leftmargin=1.5em,label={[}\arabic*{]}]
  \item H. Liu, Z. Wang, R. Li, X. Zhao, T. Xu, T. Zhou, H. Hu, ``A Comparative Study of Stereo-Dependent SSVEP Targets and Their Impact on VR-BCI Performance,'' \textit{Frontiers in Neuroscience}, vol.~18, p.~1367932, 2024. \href{https://doi.org/10.3389/fnins.2024.1367932}{doi:10.3389/fnins.2024.1367932}
  \item H. Liu, Z. Wang, R. Li, et al., ``A Novel SSVEP Modulation Method Utilizing VR-Based Binocular Vision,'' \textit{IEEE EMBC}, 2024, pp.~1--4. \href{https://doi.org/10.1109/EMBC53108.2024.10781783}{doi:10.1109/EMBC53108.2024.10781783}
  \item H. Liu, Z. Zhu, Z. Wang, et al., ``Design and Implementation of a Scalable and High-Throughput EEG Acquisition and Analysis System,'' \textit{Moore and More}, vol.~1, p.~14, 2024. \href{https://doi.org/10.1007/s44275-024-00017-w}{doi:10.1007/s44275-024-00017-w}
  \item Y. Ye, X. Tian, H. Liu, J. Liu, C. Zhou, H. Nie, W. Yu, L. Qin, Z. Zhou, X. Wei, J. Zhao, Z. Wang, M. Li, T. H. Tao, L. Sun, ``High-Precision, Low-Threshold Neuromodulation With Ultraflexible Electrode Arrays for Brain-to-Brain Interfaces,'' \textit{Exploration}, e70040, 2025. \href{https://doi.org/10.1002/EXP.70040}{doi:10.1002/EXP.70040}
  \item A. Li, H. Liu, T. Xu, T. Zhou, H. Hu, Z. Wang, ``Enhancing Real-Time Online Motor Imagery BCI Performance: A Co-Adaptive Meta-Learning Approach,'' \textit{IEEE Journal of Biomedical and Health Informatics}, revision in progress.
  \item H. Liu, Z. Wang, R. Li, X. Zhao, T. Xu, T. Zhou, H. Hu, ``A Novel Binocular-Encoded SSVEP Framework for Efficient VR-Based Brain-Computer Interface,'' \textit{IEEE Journal of Biomedical and Health Informatics}, accepted Oct.~23, 2025.
\end{enumerate}

\textbf{Patents}
\begin{itemize}
  \item National Invention Patent: Hongsong Hu; Haifeng Liu; Zhenyu Cui; Ting Zhou; Tianwei Xu; Yuling Ouyang; Yu Zhao, ``SSVEP-ERP Based BCI Visual Paradigm Generation, Detection Method, System, Medium, and Terminal,'' CN202310369762.8 (published).
  \item National Invention Patent: Hongsong Hu; Haifeng Liu; Yuan Qin; Zhenyu Wang; Ting Zhou; Tianwei Xu; Yuling Ouyang, ``Augmented-Reality Brain Feedback System Based on Brain-Computer Interfaces,'' 202410293427.9 (preliminary examination).
\end{itemize}

\sectiontitle{Selected Grants \& Collaborations}
\begin{itemize}
  \item National Natural Science Foundation (Key Area Joint Fund): Brain-Computer Communication and Brain Internet Technologies
  \item NSFC Key Program: Open-Loop Brain-Computer Interface System Techniques
  \item NSFC Young Scientists Program: Novel BCI System Technologies and Validation
  \item Shanghai Science and Technology Innovation Action (Instrumentation): High-Throughput Scalable EEG Acquisition Analyzer
  \item Shanghai STIA (MR Pre-Research): Brain Internet Key Technologies and Systems
  \item Shanghai Science and Technology Commission Key Project: Interpretable Invasive BCI Systems and Scalable Deployment
  \item Shanghai ``Tongchuang Nest'' Initiative: Multimodal Human--Computer Interface Signal Fusion Technologies
  \item Shanghai VR Innovation Program: Multidimensional Biosignal-Based Platforms for Neurological Disorder Modulation
\end{itemize}

\sectiontitle{Skills}
\begin{itemize}
  \item Academic writing expert; advanced \LaTeX{} typesetting.
  \item MATLAB and Python for signal processing, data analytics, feature extraction, and decoding.
  \item Proficient with Neuroscan and NeuroPro EEG systems plus Curry, EEGlab, Psychtoolbox, and related BCI software platforms.
  \item C++, C\#, and Unity for interactive system development; hands-on VR/AR experience with Hololens2 and PicoVR devices.
  \item Embedded development with ESP32, Nordic nRF5340, and Zephyr RTOS.
  \item Fusion 360 for 3D printing workflows and FPGA prototyping expertise.
  \item Visual/graphical programming (e.g., LabVIEW, Vive Coding) for rapid construction of BCI demos and control prototypes.
\end{itemize}

\end{document}
