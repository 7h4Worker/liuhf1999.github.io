\documentclass[11pt,a4paper]{article}
\usepackage{geometry}
\geometry{top=1.5cm,bottom=1.8cm,left=1.6cm,right=1.6cm}
\usepackage{fontspec}
\usepackage{xeCJK}
\usepackage{graphicx}
\usepackage{enumitem}
\usepackage{xcolor}
\usepackage{hyperref}
\usepackage{array}

\setmainfont{Times New Roman}
\setsansfont{Arial}
\setmonofont{Consolas}
\setCJKmainfont{SimSun}

\definecolor{Primary}{RGB}{0,118,163}
\definecolor{Accent}{RGB}{190,209,224}

\hypersetup{
  colorlinks=true,
  linkcolor=Primary,
  urlcolor=Primary
}

\setlength{\parindent}{0pt}
\setlength{\parskip}{0.3em}
\setlist[itemize]{leftmargin=1.4em,itemsep=0.3em,topsep=0.2em}

\newcommand{\sectiontitle}[1]{%
  \vspace{1.2em}%
  {\Large\bfseries\textcolor{Primary}{#1}}\par
  \textcolor{Primary}{\rule{\textwidth}{0.8pt}}\vspace{0.3em}
}

\newcommand{\timeline}[2]{%
  \textbf{#1}\quad #2\par
}


\begin{document}
\pagestyle{empty}

\begin{minipage}[t]{0.72\textwidth}
\textbf{姓\quad 名:} 刘海峰\\
\textbf{联系电话:} 173-4201-8052\\
\textbf{电子邮件:} \href{mailto:liuhf@shanghaitech.edu.cn}{liuhf@shanghaitech.edu.cn}\\
\textbf{政治面貌:} 中共党员\\
\textbf{联系地址:} 上海市浦东新区张江高科技园区海科路 99 号
\end{minipage}
\begin{minipage}[t]{0.26\textwidth}
\raggedleft
\fbox{\parbox[c][4.4cm][c]{3.5cm}{\centering 个人照片}}\\
{\small 将方框替换为证件照}
\end{minipage}

\sectiontitle{教育背景}
\begin{tabular}{@{}p{0.23\textwidth}p{0.4\textwidth}p{0.19\textwidth}p{0.15\textwidth}@{}}
2017.9--2021.7 & 浙江大学 & 信息工程 & 工学学士\\
2021.9--2026.7 & 上海科技大学(中科院联培) & 计算机科学与技术 & 工学博士(预计 2026.7 毕业)
\end{tabular}

\sectiontitle{研究方向与内容}
\textbf{研究方向:} 面向双目视觉特性的脑机接口范式与算法,构建融合双目头显设备的高效可穿戴脑机接口系统。
\begin{itemize}
  \item 开展立体依赖脑电的 SSVEP 目标比较研究,评估不同目标配置对虚拟现实脑机接口性能的影响。
  \item 提出 3D 与 3D-Blink 视觉范式,论证立体视觉在虚拟现实环境下提升用户舒适度与系统性能的可能性。
\end{itemize}

\timeline{2024.7}{面向虚拟现实双目视觉的稳态视觉诱发电位(SSVEP)调制新方法}
\begin{itemize}
  \item 提出了基于 VR 双视角的创新 SSVEP 调制方法,通过左右视图图层组合仅使用 2 个频率编码 9 个目标,显著提升系统稳健性。
  \item 设计并实现神经融合接口平台 FusionCA,打通双目头显的多路信息传输链路,改善沉浸式交互体验。
\end{itemize}

\timeline{2024.9}{可扩展高密度脑电(EEG)采集与分析系统}
\begin{itemize}
  \item 构建基于 FPGA 的高密度 EEG 采集平台,实现模块化可扩展架构,兼顾标准器件与多场景脑机接口实验需求。
\end{itemize}

\timeline{2025.1}{高精度、低阈值神经调制与超柔性电极阵列脑对脑接口}
\begin{itemize}
  \item 基于超柔性电极阵列实现脑对脑接口系统,将人脑脑机接口指令传递至小鼠,实现对其运动行为的精准控制。
  \item 构建在线信号传输与 DNN 解码模型,在仿生系统中实现人脑侧 BCI 指令映射。
\end{itemize}

\sectiontitle{项目工作}
\timeline{2022.7--2022.11}{SariBCI 脑电信号采集系统}
\begin{itemize}
  \item 以 ADS1299 为核心设计高精度脑电采集模块,实现 24-bit 精度与 250--1000 Hz 采样率,支持 WiFi AP/Station 模式切换。
  \item 基于 Unity 平台定制双视角 3D 提示界面和可视分析工具,结合 3D/3D-Blink 模式提升 SSVEP 刺激效率。
\end{itemize}

\timeline{2023.5--2023.9}{Hololens 双目 AR-BCI 闭环系统}
\begin{itemize}
  \item 构建面向 Hololens 的 UWP 应用,打通 SariBCI 采集系统到 VR/AR 设备的数据链路,实现脑控闭环。
  \item 集成 Zigbee 外设扩展多传感器节点,支持脑控多手势交互与移动部署。
\end{itemize}

\timeline{2024.3--2025.7}{多模态多传感器信号采集一体化}
\begin{itemize}
  \item 设计集成脑电、心电、肌电与惯性测量的多模态终端,通过上位机推理协调硬件联动并完成数据融合。
  \item 提出在线深度学习模型,结合时序与时空特征并部署至 FPGA,实现嵌入式边缘推理。
\end{itemize}

\timeline{2024.6--2025.9}{超宽带脑电采集与远程传输系统}
\begin{itemize}
  \item 推出多通道时域压缩与超宽带高速采集方案,构建跨模态传感器的低功耗无线脑电采集系统。
  \item 完成多通道移植与云端实时交互,建立长程脑电数据传输与管理平台。
\end{itemize}

\sectiontitle{科研发表}
\begin{enumerate}[leftmargin=1.5em,label={[}\arabic*{]}]
  \item H. Liu, Z. Wang, R. Li, X. Zhao, T. Xu, T. Zhou, H. Hu, ``A Comparative Study of Stereo-Dependent SSVEP Targets and Their Impact on VR-BCI Performance,'' \textit{Frontiers in Neuroscience}, vol.~18, p.~1367932, 2024. \href{https://doi.org/10.3389/fnins.2024.1367932}{doi:10.3389/fnins.2024.1367932}
  \item H. Liu, Z. Wang, R. Li, et al., ``A Novel SSVEP Modulation Method Utilizing VR-Based Binocular Vision,'' \textit{IEEE EMBC}, 2024, pp.~1--4. \href{https://doi.org/10.1109/EMBC53108.2024.10781783}{doi:10.1109/EMBC53108.2024.10781783}
  \item H. Liu, Z. Zhu, Z. Wang, et al., ``Design and Implementation of a Scalable and High-Throughput EEG Acquisition and Analysis System,'' \textit{Moore and More}, vol.~1, p.~14, 2024. \href{https://doi.org/10.1007/s44275-024-00017-w}{doi:10.1007/s44275-024-00017-w}
  \item Y. Ye, X. Tian, H. Liu, J. Liu, C. Zhou, H. Nie, W. Yu, L. Qin, Z. Zhou, X. Wei, J. Zhao, Z. Wang, M. Li, T. H. Tao, L. Sun, ``High-Precision, Low-Threshold Neuromodulation With Ultraflexible Electrode Arrays for Brain-to-Brain Interfaces,'' \textit{Exploration}, e70040, 2025. \href{https://doi.org/10.1002/EXP.70040}{doi:10.1002/EXP.70040}
  \item A. Li, H. Liu, T. Xu, T. Zhou, H. Hu, Z. Wang, ``Enhancing Real-Time Online Motor Imagery BCI Performance: A Co-Adaptive Meta-Learning Approach,'' \textit{IEEE Journal of Biomedical and Health Informatics}, 修回中。
  \item H. Liu, Z. Wang, R. Li, X. Zhao, T. Xu, T. Zhou, H. Hu, ``A Novel Binocular-Encoded SSVEP Framework for Efficient VR-Based Brain-Computer Interface,'' \textit{IEEE Journal of Biomedical and Health Informatics}, 已录用(2025.10.23)。
\end{enumerate}

\textbf{专利}
\begin{itemize}
  \item 【国家发明专利】胡宏松;刘海峰;崔振宇;周婷;徐天玮;欧阳玉玲;赵瑜,``基于 SSVEP-ERP 的脑机接口视觉范式生成、检测方法、系统、介质、终端'',专利号:CN202310369762.8(已公开)。
  \item 【国家发明专利】胡宏松;刘海峰;覃源;王振宇;周婷;徐天玮;欧阳玉玲,``一种基于增强现实和脑机接口的新型脑反馈系统'',专利号:202410293427.9(已初审)。
\end{itemize}

\sectiontitle{参与项目}
\begin{itemize}
  \item 国家自然科学基金重点领域联合基金重点项目——“脑机通信与脑联网关键技术研究”
  \item 国家自然科学基金重点项目——“脑机接口开环复杂系统技术研究”
  \item 国家自然科学基金青年项目——“新型脑机系统关键技术及验证”
  \item 上海市科技创新行动计划·仪器专项——“高通量可扩展脑电采集分析仪”
  \item 上海市科技创新行动计划·MR 预研——“脑联网关键技术与系统研究”
  \item 上海市科学技术委员会重点项目——“感知可解释性侵入式脑机接口系统研究及规模化应用”
  \item 上海市“同创巢”计划项目——“多模态人机接口信号融合处理关键技术研究”
  \item 上海市虚拟现实创新专项——“基于多维生物信号采集的脑疾病疗法调控平台基础研究”
\end{itemize}

\sectiontitle{个人技能}
\begin{itemize}
  \item 精通学术写作,熟悉 \LaTeX{} 排版系统。
  \item 熟练使用 MATLAB、Python 进行信号处理、数据分析、特征提取与解码。
  \item 熟悉 Neuroscan、博睿泰等脑电实验设备,掌握 Curry、EEGlab、Psychtoolbox 等脑电系统软件平台。
  \item 熟练使用 C++、C\# 与 Unity 开发交互系统,具备 Hololens2、PicoVR 等双目 VR/AR 设备开发经验。
  \item 熟悉嵌入式系统开发,尤其是 ESP32、Nordic nRF5340 与 Zephyr 平台。
  \item 能够使用 Fusion 360 进行 3D 打印建模,并具备 FPGA 项目开发经验。
  \item 掌握可视化编程(如 LabVIEW、Vive Coding),可快速搭建脑机接口展示系统与控制原型。
\end{itemize}

\end{document}
